%!TEX root = ts.tex

\rSec0[summary]{Library summary}

\begin{libreqtab2}
{Networking library summary}
{tab:summary}
\\ \topline
\lhdr{Clause}  &
\rhdr{Header(s)} \\ \capsep
\endfirsthead
\continuedcaption\\
\hline
\lhdr{Clause}  &
\rhdr{Header(s)} \\ \capsep
\endhead

Convenience header~(\ref{convenience.hdr})  &
\tcode{<experimental/net>}  \\ \rowsep

Forward declarations~(\ref{fwd.decl})  &
\tcode{<experimental/netfwd>}  \\ \rowsep

Asynchronous model~(\ref{async})  &
\tcode{<experimental/executor>}  \\ \rowsep

Basic I/O services~(\ref{io_context})  &
\tcode{<experimental/io_context>}  \\ \rowsep

Timers~(\ref{timer})  &
\tcode{<experimental/timer>}  \\ \rowsep

Buffers~(\ref{buffer})  &
\tcode{<experimental/buffer>}  \\
Buffer-oriented streams~(\ref{buffer.stream})  &
 \\ \rowsep

Sockets~(\ref{socket})  &
\tcode{<experimental/socket>}  \\
Socket iostreams~(\ref{socket.iostreams})  &
 \\
Socket algorithms~(\ref{socket.algo})  &
 \\ \rowsep

Internet protocol~(\ref{internet})  &
\tcode{<experimental/internet>}  \\

\end{libreqtab2}

\pnum
Throughout this document, where template parameters are not constrained,
the names of the template parameters are used to express type requirements,
as listed in Table~\ref{tab:summary.requirements}.
If a template parameter's name is a name in the first column,
the template argument shall meet the corresponding type requirements
in the second column.

\ednote{Many of these type requirements should be replaced by concepts,
but that will be proposed separately.}

\begin{libreqtab2}
{Template parameters and type requirements}
{tab:summary.requirements}
\\ \topline
\lhdr{template parameter name}  &
\rhdr{type requirements} \\ \capsep
\endfirsthead
\continuedcaption\\
\hline
\lhdr{template parameter name}  &
\rhdr{type requirements} \\ \capsep
\endhead

\tcode{AcceptableProtocol}  &
\newoldconcept{AcceptableProtocol}~(\ref{socket.reqmts.acceptableprotocol})  \\ \rowsep

\tcode{Allocator}  &
 \oldconcept{Allocator} (\CppXref{allocator.requirements})  \\ \rowsep

\tcode{AsyncReadStream}  &
\newoldconcept{AsyncReadStream}~(\ref{buffer.stream.reqmts.asyncreadstream})  \\ \rowsep

\tcode{AsyncWriteStream}  &
\newoldconcept{AsyncWriteStream}~(\ref{buffer.stream.reqmts.asyncwritestream})  \\ \rowsep

\tcode{Clock}  &
 \oldconcept{Clock} (\CppXref{time.clock.req})  \\ \rowsep

\tcode{CompletionCondition}  &
\newoldconcept{CompletionCondition}~(\ref{buffer.stream.reqmts.completioncondition})  \\ \rowsep

\tcode{CompletionToken}  &
completion token~(\ref{async.reqmts.async.token})  \\ \rowsep

\tcode{ConnectCondition}  &
\newoldconcept{ConnectCondition}~(\ref{socket.reqmts.connectcondition})  \\ \rowsep

\tcode{ConstBufferSequence}  &
\newoldconcept{ConstBufferSequence}~(\ref{buffer.reqmts.constbuffersequence})  \\ \rowsep

\tcode{DynamicBuffer}  &
\newoldconcept{DynamicBuffer}~(\ref{buffer.reqmts.dynamicbuffer})  \\ \rowsep

\tcode{EndpointSequence}  &
\newoldconcept{EndpointSequence}~(\ref{socket.reqmts.endpointsequence})  \\ \rowsep

\tcode{ExecutionContext}  &
\newoldconcept{ExecutionContext}~(\ref{async.reqmts.executioncontext})  \\ \rowsep

\tcode{Executor}  &
\newoldconcept{Executor}~(\ref{async.reqmts.executor})  \\ \rowsep

\tcode{GettableSocketOption}  &
\newoldconcept{GettableSocketOption}~(\ref{socket.reqmts.gettablesocketoption})  \\ \rowsep

\tcode{InternetProtocol}  &
\newoldconcept{InternetProtocol}~(\ref{internet.reqmts.protocol})  \\ \rowsep

\tcode{IoControlCommand}  &
\newoldconcept{IoControlCommand}~(\ref{socket.reqmts.iocontrolcommand})  \\ \rowsep

\tcode{MutableBufferSequence}  &
\newoldconcept{MutableBufferSequence}~(\ref{buffer.reqmts.mutablebuffersequence})  \\ \rowsep

\tcode{ProtoAllocator}  &
\newoldconcept{ProtoAllocator}~(\ref{async.reqmts.proto.allocator})  \\ \rowsep

\tcode{Protocol}  &
\newoldconcept{Protocol}~(\ref{socket.reqmts.protocol})  \\ \rowsep

\tcode{Service}  &
\newoldconcept{Service}~(\ref{async.reqmts.service})  \\ \rowsep

\tcode{SettableSocketOption}  &
\newoldconcept{SettableSocketOption}~(\ref{socket.reqmts.settablesocketoption})  \\ \rowsep

\tcode{Signature}  &
\newoldconcept{Signature}~(\ref{async.reqmts.signature})  \\ \rowsep

\tcode{SyncReadStream}  &
\newoldconcept{SyncReadStream}~(\ref{buffer.stream.reqmts.syncreadstream})  \\ \rowsep

\tcode{SyncWriteStream}  &
\newoldconcept{SyncWriteStream}~(\ref{buffer.stream.reqmts.syncwritestream})  \\ \rowsep

\tcode{WaitTraits}  &
\newoldconcept{WaitTraits}~(\ref{timer.reqmts.waittraits})  \\

\end{libreqtab2}


