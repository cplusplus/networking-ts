%!TEX root = ts.tex

\rSec0[fwd.decl]{Forward declarations}


\indexlibrary{\idxhdr{experimental/netfwd}}%
\rSec1[fwd.decl.synop]{Header \tcode{<experimental/netfwd>} synopsis}

\begin{codeblock}
@\removed{namespace std \{}@
@\removed{namespace experimental \{ }@
@\removed{namespace net \{ }@
@\removed{inline }@namespace @\added{std::experimental::net::inline}@ @\namespacever@ {

  class execution_context;
  template<class T, class Executor>
    class executor_binder;
  template<class Executor>
    class executor_work_guard;
  class system_executor;
  class executor;
  template<class Executor>
    class strand;

  class io_context;

  template<class Clock> struct wait_traits;
  template<class Clock, class WaitTraits = wait_traits<Clock>>
    class basic_waitable_timer;
  using system_timer = basic_waitable_timer<chrono::system_clock>;
  using steady_timer = basic_waitable_timer<chrono::steady_clock>;
  using high_resolution_timer = basic_waitable_timer<chrono::high_resolution_clock>;

  template<class Protocol>
    class basic_socket;
  template<class Protocol>
    class basic_datagram_socket;
  template<class Protocol>
    class basic_stream_socket;
  template<class Protocol>
    class basic_socket_acceptor;
  template<class Protocol, class Clock = chrono::steady_clock,
    class WaitTraits = wait_traits<Clock>>
      class basic_socket_streambuf;
  template<class Protocol, class Clock = chrono::steady_clock,
    class WaitTraits = wait_traits<Clock>>
      class basic_socket_iostream;

  namespace ip {

    class address;
    class address_v4;
    class address_v6;
    template<class Address>
      class basic_address_iterator;
    using address_v4_iterator = basic_address_iterator<address_v4>;
    using address_v6_iterator = basic_address_iterator<address_v6>;
    template<class Address>
      class basic_address_range;
    using address_v4_range = basic_address_range<address_v4>;
    using address_v6_range = basic_address_range<address_v6>;
    class network_v4;
    class network_v6;
    template<class InternetProtocol>
      class basic_endpoint;
    template<class InternetProtocol>
      class basic_resolver_entry;
    template<class InternetProtocol>
      class basic_resolver_results;
    template<class InternetProtocol>
      class basic_resolver;
    class tcp;
    class udp;

  } // namespace ip
} // inline namespace \added{std::experimental::net::}\namespacever
@\removed{ \} // namespace net}@
@\removed{ \} // namespace experimental}@
@\removed{ \} // namespace std}@
\end{codeblock}

\pnum
Default template arguments are described as appearing both in \tcode{<netfwd>}
and in the synopsis of other headers but it is well-formed to include both
\tcode{<netfwd>} and one or more of the other headers.
\begin{note} It is the implementation's responsibility to implement headers so
that including \tcode{<netfwd>} and other headers does not violate the rules
about multiple occurrences of default arguments. \end{note}



