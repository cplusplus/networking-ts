%!TEX root = ts.tex
\begingroup
\def\hd{\begin{tabular}{ll}
          \textbf{Document Number:} & {\larger\docno}             \\
          \textbf{Date:}            & \reldate                    \\
          \textbf{Audience:}        & LWG                         \\
          \textbf{Reply to:}        & Jonathan Wakely             \\
                                    & cxx@kayari.org
          \end{tabular}
}
\newlength{\hdwidth}
\settowidth{\hdwidth}{\hd}
\hfill\begin{minipage}{\hdwidth}\hd\end{minipage}
\endgroup

\vspace{2.5cm}
\begin{center}
\textbf{\Huge
\doctitle}
\end{center}
\vspace{2.5cm}

This paper updates the Networking TS working draft (N4771)
to use \Cpp{}20 as its base document, rather than \Cpp{}14.

The main changes proposed are:
\begin{itemize}
\item Conform to current LWG conventions
  such as using the \mandates and \constraints elements.
\item Replace type requirements like \tcode{MoveConstructible} with
  \oldconcept{MoveConstructible} and use preferred terminology (``meets'').
\item Use consistent formatting for type requirements, to distinguish them
  from template parameters and other identifiers.
\item Remove redundant \tcode{operator==} and \tcode{operator!=} overloads
  which can be synthesized by the compiler.
  Replace \tcode{operator<} etc. with \tcode{operator<=>}.
\item Use \tcode{_v} variable templates for traits.
\item Remove reference to Library Fundamentals TS, as \tcode{string_view}
  is in the IS now.
\end{itemize}

The changes are presented as \added{additions} and \removed{removals}
with the entire paper as context, even the parts with no changes.
Updated cross-references of the form \CppXIV{}[structure.specifications] 
are shown as \CppXref{structure.specifications} without repeating
\removed{2014} every time.

After applying these changes, the editor should reorder all new
\constraints and \mandates elements to match the conventional order.

\vspace{2.5cm}

Thanks to Tomasz Kamiński and Frank Birbacher for their detailed reviews of R0.
Thanks to Tim Song for reviewing R1.

Changes since R1:
\begin{itemize}
\item Change "signature" to "call signature" in [structure.specifications].
\item Removed "shall" uses in [err.report.async].
\item Replaced \exposid{DECAY_COPY} with \exposid{decay-copy} everywhere
  (and adjusted cross references from [thread.decay.copy] to [expos.only.func]).
\item Restored \expects in [basic.strand.assign] in addition to \constraints{}.
\item Reverted change to make IP address ranges satisfy borrowed view concept.
\item Fixed \textit{Mandates} for \tcode{executor_binder} constructors.
\item Restored ordering between v4 and v6 IP addresses.
\end{itemize}

Changes since R0:
\begin{itemize}
\item All uses of \tcode{decay_t} reviewed and changed to
  \tcode{remove_cvref_t} where appropriate.
\item Changed "signature" to "call signature" in [structure.specification].
\item The second column in Table~\ref{tab:summary.requirements} now uses
  an italicized \textit{ConceptName} for all rows except one.
\item Added \tcode{explicit} default constructor to \tcode{executor_arg_t}.
\item Restored some "shall" requirements on users that should not have been removed.
\item Make IP address ranges satisfy borrowed view concept.
\end{itemize}

Future work:
\begin{itemize}
\item
Replace non-member operators with hidden friends.
LWG would prefer that to be done in a separate proposal to go via LEWG.
\item
Use concepts to replace \newoldconcept{TypeRequirements}.
\item
Make address ranges satisfy range concepts (currently not possible as
their iterators don't satisfy \tcode{std::input_or_output_iterator}).
\item
Try to specify "completion token" as proper type requirements (or a concept),
unless SG4 are replacing that part of the spec.
\item
Consider replacing \exposid{DEDUCED} with an exposition-only alias template,
e.g. \exposid{async-return-type}\tcode{<CompletionToken, Signature>}
or similar.
\end{itemize}

Open questions (probably to be filed as LWG issues):
\begin{itemize}
\item
Clarify intent of \tcode{strand} move constructor and move assignment.
\item
Should \tcode{address_v4} comparisons really use \tcode{to_uint()},
so that whether 1.0.0.4 is less than 2.0.0.3 depends on host endianness?
\end{itemize}


\newpage

%%--------------------------------------------------
%% The table of contents
% \markboth{\contentsname}{}

%% Include table of contents. Do not list "Contents"
%% within it (per ISO request) but do include a
%% bookmark for it in the PDF.
\phantomsection
\pdfbookmark{\contentsname}{toctarget}
\hypertarget{toctarget}{\tableofcontents*}

\setcounter{tocdepth}{5}

%%!TEX root = std.tex
\chapter{Foreword}

%ISO (the International Organization for Standardization)
%is a worldwide federation of national standards bodies (ISO member bodies).
%The work of preparing International Standards
%is normally carried out through ISO technical committees.
%Each member body interested in a subject
%for which a technical committee has been established
%has the right to be represented on that committee.
%International organizations,
%governmental and non-governmental,
%in liaison with ISO,
%also take part in the work.
%ISO collaborates closely with
%the International Electrotechnical Commission (IEC)
%on all matters of electrotechnical standardization.

ISO (the International Organization for Standardization) and IEC (the
International Electrotechnical Commission) form the specialized system for
worldwide standardization. National bodies that are members of ISO or IEC
participate in the development of International Standards through technical
committees established by the respective organization to deal with particular
fields of technical activity. ISO and IEC technical committees collaborate in
fields of mutual interest. Other international organizations, governmental and
non-governmental, in liaison with ISO and IEC, also take part in the work. In
the field of information technology, ISO and IEC have established a joint
technical committee, ISO/IEC JTC 1.

The procedures used to develop this document and those intended for its further
maintenance are described in the ISO/IEC Directives, Part 1. In particular the
different approval criteria needed for the different types of ISO documents should
be noted. This document was drafted in accordance with the editorial rules of
the ISO/IEC Directives, Part 2
(see \href{http://www.iso.org/directives}{\tcode{www.iso.org/directives}}).

Attention is drawn to the possibility that some of the elements of this
document may be the subject of patent rights. ISO shall not be held
responsible for identifying any or all such patent rights. Details of any
patent rights identified during the development of the document will be in the
Introduction and/or on the ISO list of patent declarations received
(see \href{http://www.iso.org/patents}{\tcode{www.iso.org/patents}}).

Any trade name used in this document is information given for the convenience
of users and does not constitute an endorsement.

For an explanation on
the voluntary nature of standards,
the meaning of ISO specific terms and expressions related
to conformity assessment, as well as information about ISO's adherence
to the World Trade Organization (WTO) principles
in the Technical Barriers to Trade (TBT) see the following URL:
\href{http://www.iso.org/iso/foreword.html}{\tcode{www.iso.org/iso/foreword.html}}.

This document was prepared by
Technical Committee ISO/IEC JTC 1, \textit{Information technology},
Subcommittee SC 22, \textit{Programming languages, their environments and system software interfaces}.

Any feedback or questions on this document
should be directed to the user's national standards body.
A complete listing of these bodies can be found at
\href{http://www.iso.org/members.html}{\tcode{www.iso.org/members.html}}.

