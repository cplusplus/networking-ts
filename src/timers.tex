%!TEX root = ts.tex

\rSec0[timer]{Timers}

\pnum
This clause defines components for performing timer operations.

\pnum
\begin{example} Performing a synchronous wait operation on a timer:
\begin{codeblock}
io_context c;
steady_timer t(c);
t.expires_after(seconds(5));
t.wait();
\end{codeblock}
 \end{example}

\pnum
\begin{example} Performing an asynchronous wait operation on a timer:
\begin{codeblock}
void handler(error_code ec) { ... }
...
io_context c;
steady_timer t(c);
t.expires_after(seconds(5));
t.async_wait(handler);
c.run();
\end{codeblock}
 \end{example}


\indexlibrary{\idxhdr{experimental/timer}}%
\rSec1[timer.synop]{Header \tcode{<experimental/timer>} synopsis}

\begin{codeblock}
#include <chrono>

namespace std::experimental::net::inline @\namespacever@ {

  template<class Clock> struct wait_traits;

  template<class Clock, class WaitTraits = wait_traits<Clock>@\addedcode{, class Executor = executor}@>
    class basic_waitable_timer;

  using system_timer = basic_waitable_timer<chrono::system_clock>;
  using steady_timer = basic_waitable_timer<chrono::steady_clock>;
  using high_resolution_timer = basic_waitable_timer<chrono::high_resolution_clock>;

} // inline namespace std::experimental::net::\namespacever
\end{codeblock}



\rSec1[timer.reqmts]{Requirements}


\rSec2[timer.reqmts.waittraits]{Wait traits requirements}

\pnum
The \tcode{basic_waitable_timer} template uses wait traits to allow programs to customize \tcode{wait} and \tcode{async_wait} behavior.
\begin{note} Possible uses of wait traits include:
\begin{itemize}
\item
 To enable timers based on non-realtime clocks.
\item
 Determining how quickly wallclock-based timers respond to system time changes.
\item
 Correcting for errors or rounding timeouts to boundaries.
\item
 Preventing duration overflow. That is, a program can set a timer's expiry \tcode{e} to be \tcode{Clock::max()} (meaning never reached) or \tcode{Clock::min()} (meaning always in the past). As a result, computing the duration until timer expiry as \tcode{e - Clock::now()} can cause overflow. \end{itemize}\end{note}

\pnum
For a type \tcode{Clock} meeting the \oldconcept{Clock} requirements~(\CppXref{time.clock.req}), a type \tcode{X} meets the \newoldconcept{WaitTraits} requirements if it meets the requirements listed below.

\pnum
In Table~\ref{tab:timer.reqmts.waittraits.requirements},
\tcode{t} denotes a  value of type (possibly const) \tcode{Clock::time_point};
and \tcode{d} denotes a  value of type (possibly const) \tcode{Clock::duration}.

\indextext{requirements!\idxnewoldconcept{WaitTraits}}%
\begin{libreqtab3}
{\newoldconcept{WaitTraits} requirements}
{tab:timer.reqmts.waittraits.requirements}
\\ \topline
\lhdr{expression}  &
\chdr{return type}  &
\rhdr{assertion/note pre/post-condition} \\ \capsep
\endfirsthead
\continuedcaption\\
\hline
\lhdr{expression}  &
\chdr{return type}  &
\rhdr{assertion/note pre/post-condition} \\ \capsep
\endhead

\tcode{X::to_wait_duration(d)}  &
\tcode{Clock::duration}  &
Returns a \tcode{Clock::duration} value to be used in a \tcode{wait} or \tcode{async_wait} operation. \begin{note} The return value is typically representative of the duration \tcode{d}. \end{note}  \\ \rowsep

\tcode{X::to_wait_duration(t)}  &
\tcode{Clock::duration}  &
Returns a \tcode{Clock::duration} value to be used in a \tcode{wait} or \tcode{async_wait} operation. \begin{note} The return value is typically representative of the duration from \tcode{Clock::now()} until the time point \tcode{t}. \end{note}  \\

\end{libreqtab3}


\rSec1[timer.waittraits]{Class template \tcode{wait_traits}}

\indexlibrary{\idxcode{wait_traits}}%
\begin{codeblock}
namespace std::experimental::net::inline @\namespacever@ {

  template<class Clock>
  struct wait_traits
  {
    static typename Clock::duration to_wait_duration(
      const typename Clock::duration& d);

    static typename Clock::duration to_wait_duration(
      const typename Clock::time_point& t);
  };

} // inline namespace std::experimental::net::\namespacever
\end{codeblock}

\pnum
Class template \tcode{wait_traits} meets the \newoldconcept{WaitTraits}~(\ref{timer.reqmts.waittraits}) type requirements. Template argument \tcode{Clock} is a type meeting the \oldconcept{Clock} requirements (\CppXref{time.clock.req}).

\indexlibrarymember{to_wait_duration}{wait_traits}%
\begin{itemdecl}
static typename Clock::duration to_wait_duration(
  const typename Clock::duration& d);
\end{itemdecl}

\begin{itemdescr}
\pnum
\returns \tcode{d}.
\end{itemdescr}

\indexlibrarymember{to_wait_duration}{wait_traits}%
\begin{itemdecl}
static typename Clock::duration to_wait_duration(
  const typename Clock::time_point& t);
\end{itemdecl}

\begin{itemdescr}
\pnum
\returns Let \tcode{now} be \tcode{Clock::now()}. If \tcode{now + Clock::duration::max()} is before \tcode{t}, \tcode{Clock::dur\-ation::max()}; if \tcode{now + Clock::duration::min()} is after \tcode{t}, \tcode{Clock::duration::min()}; otherwise, \tcode{t - now}.
\end{itemdescr}


\rSec1[timer.waitable]{Class template \tcode{basic_waitable_timer}}

\indexlibrary{\idxcode{basic_waitable_timer}}%
\begin{codeblock}
namespace std::experimental::net::inline @\namespacever@ {

  template<class Clock, class WaitTraits = wait_traits<Clock>@\addedcode{, class Executor = executor}@>
  class basic_waitable_timer
  {
  public:
    // types:

    using executor_type = @\removedcode{io_context::executor_type}\addedcode{Executor}@;
    using clock_type = Clock;
    using duration = typename clock_type::duration;
    using time_point = typename clock_type::time_point;
    using traits_type = WaitTraits;

    @\addedcode{template<class OtherExecutor>}@
    @\addedcode{using rebind_executor =}@
    @\addedcode{  basic_waitable_timer<Clock, WaitTraits, OtherExecutor>;}@

    // \ref{timer.waitable.cons}, construct / copy / destroy

    explicit basic_waitable_timer(@\removedcode{io_context\& ctx}\addedcode{const executor_type\& ex}@);
    @\addedcode{template<class ExecutionContext>}@
    @\addedcode{  explicit basic_waitable_timer(ExecutionContext\& ctx);}@
    basic_waitable_timer(@\removedcode{io_context\& ctx}\addedcode{const executor_type\& ex}@, const time_point& t);
    @\addedcode{template<class ExecutionContext>}@
    @\addedcode{  basic_waitable_timer(ExecutionContext\& ctx, const time_point\& t);}@
    basic_waitable_timer(@\removedcode{io_context\& ctx}\addedcode{const executor_type\& ex}@, const duration& d);
    @\addedcode{template<class ExecutionContext>}@
    @\addedcode{  basic_waitable_timer(ExecutionContext\& ctx, const duration\& d);}@
    basic_waitable_timer(const basic_waitable_timer&) = delete;
    basic_waitable_timer(basic_waitable_timer&& rhs);

    ~basic_waitable_timer();

    basic_waitable_timer& operator=(const basic_waitable_timer&) = delete;
    basic_waitable_timer& operator=(basic_waitable_timer&& rhs);

    // \ref{timer.waitable.ops}, basic_waitable_timer operations

    executor_type get_executor() noexcept;

    size_t cancel();
    size_t cancel_one();

    time_point expiry() const;
    size_t expires_at(const time_point& t);
    size_t expires_after(const duration& d);

    void wait();
    void wait(error_code& ec);

    template<class CompletionToken>
      @\DEDUCED@ async_wait(CompletionToken&& token);
  };

} // inline namespace std::experimental::net::\namespacever
\end{codeblock}

\pnum
Specializations of class template \tcode{basic_waitable_timer} meet the requirements of \oldconcept{Destructible}~(\CppXref{destructible}), \oldconcept{MoveConstructible}~(\CppXref{moveconstructible}), and \oldconcept{MoveAssignable}~(\CppXref{moveassignable}).


\rSec2[timer.waitable.cons]{\tcode{basic_waitable_timer} constructors}

\indexlibrary{\idxcode{basic_waitable_timer}!constructor}%
\begin{itemdecl}
explicit basic_waitable_timer(@\removedcode{io_context\& ctx}\addedcode{const executor_type\& ex}@);
\end{itemdecl}

\begin{itemdescr}
\pnum
\effects Equivalent to \tcode{basic_waitable_timer(\removed{ctx}\added{ex}, time_point())}.
\end{itemdescr}

\begin{itemdecl}
@\addedcode{template<class ExecutionContext>}@
@\addedcode{  explicit basic_waitable_timer(ExecutionContext\& ctx);}@
\end{itemdecl}

\begin{itemdescr}
\addedpnum
\added{\effects Equivalent to \tcode{basic_waitable_timer(ctx.get_executor())}.}

\addedpnum
\added{\remarks This function shall not participate in overload resolution unless
\tcode{is_convertible<Exec\-ution\-Context\&, execution_context\&>::value} is \tcode{true}
and \tcode{is_constructible<executor_type, typename Exec\-ution\-Context::executor_type>::value} is \tcode{true}.}
\end{itemdescr}

\begin{itemdecl}
basic_waitable_timer(@\removedcode{io_context\& ctx}\addedcode{const executor_type\& ex}@, const time_point& t);
\end{itemdecl}

\begin{itemdescr}
\pnum
\postconditions
\begin{itemize}
\item
\tcode{get_executor() == \removed{ctx.get_executor()}\added{ex}}.
\item
\tcode{expiry() == t}.
\end{itemize}
\end{itemdescr}

\begin{itemdecl}
@\addedcode{template<class ExecutionContext>}@
@\addedcode{  basic_waitable_timer(ExecutionContext\& ctx, const time_point\& t);}@
\end{itemdecl}

\begin{itemdescr}
\addedpnum
\added{\effects Equivalent to \tcode{basic_waitable_timer(ctx.get_executor(), t)}.}

\addedpnum
\added{\remarks This function shall not participate in overload resolution unless
\tcode{is_convertible<Exec\-ution\-Context\&, execution_context\&>::value} is \tcode{true}
and \tcode{is_constructible<executor_type, typename Exec\-ution\-Context::executor_type>::value} is \tcode{true}.}
\end{itemdescr}

\begin{itemdecl}
basic_waitable_timer(@\removedcode{io_context\& ctx}\addedcode{const executor_type\& ex}@, const duration& d);
\end{itemdecl}

\begin{itemdescr}
\pnum
\effects Sets the expiry time as if by calling \tcode{expires_after(d)}.

\pnum
\postconditions \tcode{get_executor() == \removed{ctx.get_executor()}\added{ex}}.
\end{itemdescr}

\begin{itemdecl}
@\addedcode{template<class ExecutionContext>}@
@\addedcode{  basic_waitable_timer(ExecutionContext\& ctx, const duration\& d);}@
\end{itemdecl}

\begin{itemdescr}
\addedpnum
\added{\effects Equivalent to \tcode{basic_waitable_timer(ctx.get_executor(), d)}.}

\addedpnum
\added{\remarks This function shall not participate in overload resolution unless
\tcode{is_convertible<Exec\-ution\-Context\&, execution_context\&>::value} is \tcode{true}
and \tcode{is_constructible<executor_type, typename Exec\-ution\-Context::executor_type>::value} is \tcode{true}.}
\end{itemdescr}

\begin{itemdecl}
basic_waitable_timer(basic_waitable_timer&& rhs);
\end{itemdecl}

\begin{itemdescr}
\pnum
\effects Move constructs an object of class \tcode{basic_waitable_timer<Clock, WaitTraits>} that refers to the state originally represented by \tcode{rhs}.

\pnum
\postconditions
\begin{itemize}
\item
\tcode{get_executor() == rhs.get_executor()}.
\item
\tcode{expiry()} returns the same value as \tcode{rhs.expiry()} prior to the constructor invocation.
\item
\tcode{rhs.expiry() == time_point()}.
\end{itemize}
\end{itemdescr}



\rSec2[timer.waitable.dtor]{\tcode{basic_waitable_timer} destructor}

\indexlibrary{\idxcode{basic_waitable_timer}!destructor}%
\begin{itemdecl}
~basic_waitable_timer();
\end{itemdecl}

\begin{itemdescr}
\pnum
\effects Destroys the timer, canceling any asynchronous wait operations associated with the timer as if by calling \tcode{cancel()}.
\end{itemdescr}



\rSec2[timer.waitable.assign]{\tcode{basic_waitable_timer} assignment}

\indexlibrarymember{operator=}{basic_waitable_timer}%
\begin{itemdecl}
basic_waitable_timer& operator=(basic_waitable_timer&& rhs);
\end{itemdecl}

\begin{itemdescr}
\pnum
\effects Cancels any outstanding asynchronous operations associated with \tcode{*this} as if by calling \tcode{cancel()}, then moves into \tcode{*this} the state originally represented by \tcode{rhs}.

\pnum
\postconditions
\begin{itemize}
\item
\tcode{get_executor() == rhs.get_executor()}.
\item
\tcode{expiry()} returns the same value as \tcode{rhs.expiry()} prior to the assignment.
\item
\tcode{rhs.expiry() == time_point()}.
\end{itemize}

\pnum
\returns \tcode{*this}.
\end{itemdescr}



\rSec2[timer.waitable.ops]{\tcode{basic_waitable_timer} operations}

\indexlibrarymember{get_executor}{basic_waitable_timer}%
\begin{itemdecl}
executor_type get_executor() noexcept;
\end{itemdecl}

\begin{itemdescr}
\pnum
\returns The associated executor.
\end{itemdescr}

\indexlibrarymember{cancel}{basic_waitable_timer}%
\begin{itemdecl}
size_t cancel();
\end{itemdecl}

\begin{itemdescr}
\pnum
\effects Causes any outstanding asynchronous wait operations to complete. Completion handlers for canceled operations are passed an error code \tcode{ec} such that \tcode{ec == errc::operation_canceled} yields \tcode{true}.

\pnum
\returns The number of operations that were canceled.

\pnum
\remarks Does not block (\CppXref{defns.block}) the calling thread pending completion of the canceled operations.
\end{itemdescr}

\indexlibrarymember{cancel_one}{basic_waitable_timer}%
\begin{itemdecl}
size_t cancel_one();
\end{itemdecl}

\begin{itemdescr}
\pnum
\effects Causes the outstanding asynchronous wait operation that was initiated first, if any, to complete as soon as possible. The completion handler for the canceled operation is passed an error code \tcode{ec} such that \tcode{ec == errc::operation_canceled} yields \tcode{true}.

\pnum
\returns \tcode{1} if an operation was canceled, otherwise \tcode{0}.

\pnum
\remarks Does not block (\CppXref{defns.block}) the calling thread pending completion of the canceled operation.
\end{itemdescr}

\indexlibrarymember{expiry}{basic_waitable_timer}%
\begin{itemdecl}
time_point expiry() const;
\end{itemdecl}

\begin{itemdescr}
\pnum
\returns The expiry time associated with the timer, as previously set using \tcode{expires_at()} or \tcode{expires_after()}.
\end{itemdescr}

\indexlibrarymember{expires_at}{basic_waitable_timer}%
\begin{itemdecl}
size_t expires_at(const time_point& t);
\end{itemdecl}

\begin{itemdescr}
\pnum
\effects Cancels outstanding asynchronous wait operations, as if by calling \tcode{cancel()}. Sets the expiry time associated with the timer.

\pnum
\returns The number of operations that were canceled.

\pnum
\postconditions \tcode{expiry() == t}.
\end{itemdescr}

\indexlibrarymember{expires_after}{basic_waitable_timer}%
\begin{itemdecl}
size_t expires_after(const duration& d);
\end{itemdecl}

\begin{itemdescr}
\pnum
\returns \tcode{expires_at(clock_type::now() + d)}.
\end{itemdescr}

\indexlibrarymember{wait}{basic_waitable_timer}%
\begin{itemdecl}
void wait();
void wait(error_code& ec);
\end{itemdecl}

\begin{itemdescr}
\pnum
\effects Establishes the postcondition as if by repeatedly blocking the calling thread (\CppXref{defns.block}) for the relative time produced by \tcode{WaitTraits::to_wait_duration(expiry())}.

\pnum
\postconditions \tcode{ec || expiry() <= clock_type::now()}.
\end{itemdescr}

\indextext{asynchronous wait operation}%
\indexlibrarymember{async_wait}{basic_waitable_timer}%
\begin{itemdecl}
template<class CompletionToken>
  @\DEDUCED@ async_wait(CompletionToken&& token);
\end{itemdecl}

\begin{itemdescr}
\pnum
\completionsig \tcode{void(error_code ec)}.

\pnum
\effects Initiates an asynchronous wait operation to repeatedly wait for the relative time produced by \tcode{WaitTraits::to_wait_duration(e)}, where \tcode{e} is a value of type \tcode{time_point} such that \tcode{e <= expiry()}. The completion handler is submitted for execution only when the condition \tcode{ec || expiry() <= clock_type::now()} yields \tcode{true}.

\pnum
\begin{note} To implement \tcode{async_wait}, an \tcode{io_context} object \tcode{ctx} could maintain a priority queue for each specialization of \tcode{basic_waitable_timer<Clock, WaitTraits>} for which a timer object was initialized with \tcode{ctx}. Only the time point \tcode{e} of the earliest outstanding expiry need be passed to \tcode{WaitTraits::to_wait_duration(e)}. \end{note}
\end{itemdescr}




